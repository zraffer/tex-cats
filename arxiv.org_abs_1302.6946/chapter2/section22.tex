\documentclass[../main/CT4S-EN-RU]{subfiles}

\begin{document}

\section{\caseENGRUS{Commutative diagrams}{ / }{Коммутативные диаграммы}}\label{sec:comm diag}
\addtocounter{subsection}{1}\setcounter{subsubsection}{0}

\begin{blockENG}
At this point it is difficult to precisely define diagrams or commutative diagrams in general, but we can give the heuristic idea.%
\footnote{We will define commutative diagrams precisely in Section~\ref{sec:diagrams in a category}.}
Consider the following picture: 
\begin{align}\label{dia:triangle}
\xymatrix{A\ar[r]^f\ar[rd]_h&B\ar[d]^g\\&C}
\end{align}
We say this is a {\em diagram of sets}\index{diagram!in $\Set$} if each of $A,B,C$ is a set and each of $f,g,h$ is a function. We say this diagram {\em commutes}\index{commuting diagram}\index{diagam!commutes} if $g\circ f = h.$ In this case we refer to it as a commutative triangle of sets.
\end{blockENG}

\begin{blockRUS}
На данном этапе изложения трудно дать точное определение диаграммам или коммутативным диаграммам в общем виде, поэтому мы ограничимся пока эвристической идеей.%
\footnote{Позднее, в Разделе~\ref{sec:diagrams in a category} мы точно определим понятие коммутативной диаграммы.}
Рассмотрим следующий рисунок: 
\begin{align}\label{dia:triangle}
\xymatrix{A\ar[r]^f\ar[rd]_h&B\ar[d]^g\\&C}
\end{align}
Мы называем его {\em диаграммой множеств}\index{диаграмма!в $\Set$}, если каждое из $A,B,C$ является множеством и каждое из $f,g,h$ — функцией. Мы называем эту диаграмму {\em коммутативной}\index{коммутативная диаграмма}\index{диаграмма!коммутирует}, если $g\circ f = h.$ В данном случае мы называем ее коммутативным треугольником множеств.
\end{blockRUS}

\begin{applicationENG}
\href{http://en.wikipedia.org/wiki/Central_dogma_of_molecular_biology}{\text The central dogma of molecular biology} is that “DNA codes for RNA codes for protein”. That is, there is a function from DNA triplets to RNA triplets and a function from RNA triplets to amino acids. But sometimes we just want to discuss the translation from DNA to amino acids, and this is the composite of the other two. The commutative diagram is a picture of this fact.
\end{applicationENG}

\begin{applicationRUS}
\href{https://ru.wikipedia.org/wiki/%D0%A6%D0%B5%D0%BD%D1%82%D1%80%D0%B0%D0%BB%D1%8C%D0%BD%D0%B0%D1%8F_%D0%B4%D0%BE%D0%B3%D0%BC%D0%B0_%D0%BC%D0%BE%D0%BB%D0%B5%D0%BA%D1%83%D0%BB%D1%8F%D1%80%D0%BD%D0%BE%D0%B9_%D0%B1%D0%B8%D0%BE%D0%BB%D0%BE%D0%B3%D0%B8%D0%B8}{\text Центральная догма молекулярной биологии} это «ДНК кодирует РНК, РНК кодирует белки». То есть, имеется функция из триплетов ДНК в триплеты РНК, а также функция из триплетов РНК в аминокислоты. Но иногда мы хотим обсуждать только трансляцию [биологический термин] из ДНК в аминокислоты, и эта функция является композицией первых двух. Данная коммутативная диаграмма является графическим изображением этого факта.
\end{applicationRUS}

\begin{blockENG}
Consider the following picture:
$$\xymatrix{A\ar[r]^f\ar[d]_h&B\ar[d]^g\\C\ar[r]_i&D}$$
We say this is a {\em diagram of sets} if each of $A,B,C,D$ is a set and each of $f,g,h,i$ is a function. We say this diagram {\em commutes} if $g\circ f=i\circ h.$ In this case we refer to it as a commutative square of sets.
\end{blockENG}

\begin{blockRUS}
Рассмотрим следующий рисунок:
$$\xymatrix{A\ar[r]^f\ar[d]_h&B\ar[d]^g\\C\ar[r]_i&D}$$
Мы называем его {\em диаграммой множеств}, если каждое из $A,B,C,D$ является множеством и каждая из $f,g,h,i$ — функцией. Мы говорим, что эта диаграмма {\em коммутирует}, если $g\circ f=i\circ h.$ В данном случае мы называем это коммутативным квадратом множеств.
\end{blockRUS}

\begin{applicationENG}
Given a physical system $S,$ there may be two mathematical approaches $f\taking S\to A$ and $g\taking S\to B$ that can be applied to it. Either of those results in a prediction of the same sort, $f'\taking A\to P$ and $g'\taking B\to P.$ For example, in \href{http://en.wikipedia.org/wiki/Hamiltonian_mechanics#As_a_reformulation_of_Lagrangian_mechanics}{\text mechanics} we can use either Lagrangian approach or the Hamiltonian approach to predict future states. To say that the diagram 
$$
\xymatrix{S\ar[r]\ar[d]&A\ar[d]\\B\ar[r]&P}
$$
commutes would say that these approaches give the same result.
\end{applicationENG}

\begin{applicationRUS}
Для данной физической системы $S$ возможны два применимых к ней математических подхода $f\taking S\to A$ и $g\taking S\to B.$ Каждый из них приводит к предсказанию одинакового вида, $f'\taking A\to P$ и $g'\taking B\to P.$ Например, в \href{https://ru.wikipedia.org/wiki/%D0%93%D0%B0%D0%BC%D0%B8%D0%BB%D1%8C%D1%82%D0%BE%D0%BD%D0%BE%D0%B2%D0%B0_%D0%BC%D0%B5%D1%85%D0%B0%D0%BD%D0%B8%D0%BA%D0%B0#.D0.9F.D0.B5.D1.80.D0.B5.D1.84.D0.BE.D1.80.D0.BC.D1.83.D0.BB.D0.B8.D1.80.D0.BE.D0.B2.D0.BA.D0.B0_.D0.BB.D0.B0.D0.B3.D1.80.D0.B0.D0.BD.D0.B6.D0.B5.D0.B2.D0.BE.D0.B9_.D0.BC.D0.B5.D1.85.D0.B0.D0.BD.D0.B8.D0.BA.D0.B8}{\text механике} мы можем использовать для предсказания будущего состояния [фмзической системы] либо Лагранжев формализм, либо Гамильтонов формализм. Утверждать, что диаграмма 
$$
\xymatrix{S\ar[r]\ar[d]&A\ar[d]\\B\ar[r]&P}
$$
коммутирует, означает утверждать, что эти подходы приводят к одинаковым результатам.
\end{applicationRUS}

\begin{blockENG}
And so on. Note that diagram (\ref{dia:triangle}) is considered to be the same diagram as each of the following:
$$
\xymatrix{A\ar[r]^f\ar[d]_h&B\ar[dl]^g\\C}\hspace{.8in}
\xymatrix{A\ar[r]^f\ar@/_1pc/[rr]_h&B\ar[r]^g&C}\hspace{.8in}
\xymatrix{B\ar[rd]^g\\&C\\A\ar[ru]_h\ar[uu]^f}$$
\end{blockENG}

\begin{blockRUS}
И так далее. Заметим, что диаграмма (\ref{dia:triangle}) считается совпадающей с каждой из приведенных ниже:
$$
\xymatrix{A\ar[r]^f\ar[d]_h&B\ar[dl]^g\\C}\hspace{.8in}
\xymatrix{A\ar[r]^f\ar@/_1pc/[rr]_h&B\ar[r]^g&C}\hspace{.8in}
\xymatrix{B\ar[rd]^g\\&C\\A\ar[ru]_h\ar[uu]^f}$$
\end{blockRUS}

\end{document}
