\documentclass[../main/CT4S-EN-RU]{subfiles}

\begin{document}

\section{\caseENGRUS{A brief history of category theory}{ / }{Краткая история теории категорий}}

\begin{blockENG}
The paradigm shift brought on by Einstein's theory of relativity brought on the realization that there is no single perspective from which to view the world. There is no background framework that we need to find; there are infinitely many different frameworks and perspectives, and the real power lies in being able to translate between them. It is in this historical context that category theory got its start.%
\footnote{The following history of category theory is far too brief, and perhaps reflects more of the author's aesthetic than any kind of objective truth, whatever that may mean. Here are some much better references: \cite{Kro}, \cite{Mar1}, \cite{LM}.}
\end{blockENG}


\begin{blockRUS}
Сдвиг парадигмы, произведенный теорией относительности Эйнштейна, принес понимание того, что нет единой точки зрения на мир. Не существует основополагающей системы отсчёта, которую нам нужно найти; имеется бесконечно много различных систем отсчета и точек зрения, и настоящая мощь состоит в умении переходить между ними. В таком историческом контексте возникала теория категорий.%
\footnote{Нижеследующее описание истории теории категорий слишком кратко, и, возможно, отражает в большей степени авторское чувство прекрасного, чем любой вид объективной истины, что бы это ни значило. Вот некоторые значительно лучшие ссылки: \cite{Kro}, \cite{Mar1}, \cite{LM}.}%
\endnote{
TODO ...(дать краткий обзор областей теории категорий в исторической перспективе по темам: функторы, сопряжения, монады, алгебраические теории с категорной точки зрения, топосы как пример последних, теория высших категорий (гомотопии, кобордизмы, категория категорий), HoTT Воеводского)...
}
\end{blockRUS}

\begin{blockENG}
Category theory was invented in the early 1940s by Samuel Eilenberg\index{Eilenberg, Samuel} and Saunders Mac Lane.\index{Mac Lane, Saunders} It was specifically designed to bridge what may appear to be two quite different fields: topology and algebra. Topology is the study of abstract shapes such as 7-dimensional spheres; algebra is the study of abstract equations such as $y^2z=x^3-xz^2.$ People had already created important and useful links (e.g. cohomology theory) between these fields, but Eilenberg and Mac Lane needed to precisely compare different links with one another. To do so they first needed to boil down and extract the fundamental nature of these two fields. But the ideas they worked out amounted to a framework that fit not only topology and algebra, but many other mathematical disciplines as well.
\end{blockENG}

\begin{blockRUS}
Теория категорий была создана в начале 1940-х Сэмюэлем Эйленбергом\index{Эйленберг, Сэмюэль} и Сондерсом Мак Лейном\index{Мак Лейн, Сондерс}. Она специально разрабатывалась для того, чтобы соединить две, на первый взгляд, совершенно различные области: топологию и алгебру. Топология изучает абстрактные формы, такие как 7-мерные сферы; алгебра изучает абстрактные уравнения, такие как $y^2z=x^3-xz^2.$ Математики уже создали к тому времени важные и полезные связи (например, теорию когомологий) между этими областями, но Эйленбергу и Мак Лейну потребовалось провести точное сравнение различных связей друг с другом. Для этого им сперва потребовалось «выпарить и экстрагировать» фундаментальную природу этих двух областей. Однако идеи, на которых они проводили первые испытания, привели их к конструкции, которая подходила не только для топологии и алгебры, но с равным успехом была применима и для многих других математических дисциплин. 
\end{blockRUS}

\begin{blockENG}
At first category theory was little more than a deeply clarifying language for existing difficult mathematical ideas. However, in 1957 Alexander Grothendieck\index{Grothendieck!in history} used category theory to build new mathematical machinery (new cohomology theories) that granted unprecedented insight into the behavior of algebraic equations. Since that time, categories have been built specifically to zoom in on particular features of mathematical subjects and study them with a level of acuity that is simply unavailable elsewhere.
\end{blockENG}

\begin{blockRUS}
В самом начале теория категорий была не более, чем глубоко проясняющим языком для отдельных уже существующих трудных математических идей. Однако, в 1957 году Александр Гротендик\index{Гротендик!в истории} использовал теорию категорий для построения новой математической техники (новых когомологических теорий), которая предоставила беспрецендентное понимание поведения алгебраических уравнений. С тех пор категории стали строить специально для того, чтобы сосредоточиться на отдельных свойствах математических объектов и изучать их с такой четкостью, которая была ранее просто недоступна. 
\end{blockRUS}

\begin{blockENG}
Bill Lawvere\index{Lawvere, William} saw category theory as a new foundation for all mathematical thought. Mathematicians had been searching for foundations in the 19th century and were reasonably satisfied with set theory as {\em the foundation}. But Lawvere showed that the category of sets is simply a category with certain nice properties, not necessarily the center of the mathematical universe. He explained how whole algebraic theories can be viewed as examples of a single system. He and others went on to show that higher order logic was beautifully captured in the setting of category theory (more specifically toposes). It is here also that Grothendieck and his school worked out major results in algebraic geometry.
\end{blockENG}

\begin{blockRUS}
Билл Ловер\index{Ловер, Уилльям} увидел в теории категорий новые основания для всех математических построений. В поисках оснований математики находились еще в 19-м веке и ко временам Ловера были в достаточной степени удовлетворены использованием теории множеств в качестве {\em оснований}.%
\endnote{
TODO ...(неплохо бы объяснить читателю, что на математическом жаргоне называют «основаниями»)...
} Однако Ловер показал, что категория множеств это просто категория с определенными хорошими свойствами, а не обязательный центр математической вселенной. Он показал, как целые алгебраические теории могут рассматриваться в качестве примеров единой системы. Ему и другим удалось показать, что логика высших порядков прекрасно охватывается средствами теории категорий (конкретно, топосов). Пользуясь инструментами из этой же области, Гротендику и его школе удалось получить значительные результаты в алгебраической геометрии. 
\end{blockRUS}

\begin{blockENG}
In 1980 Joachim Lambek\index{Lambek, Joachim} showed that the types and programs used in computer science form a specific kind of category. This provided a new semantics for talking about programs, allowing people to investigate how programs combine and compose to create other programs, without caring about the specifics of implementation. Eugenio Moggi\index{Moggi, Eugenio} brought the category theoretic notion of monads into computer science to encapsulate ideas that up to that point were considered outside the realm of such theory.
\end{blockENG}

\begin{blockRUS}
В 1980 Иоахим Ламбек\index{Ламбек, Иоахим} показал, что типы и программы, используемые в информатике, образуют особого рода категорию. Это дало возможность обсуждать программы в рамках новой семантики, позволяя ученым исследовать, как программы комбинируются и соединяются в другие программы, не заботясь о специфике реализации. Эугенио Моджи\index{Моджи, Эугенио} принес в информатику теоретико-категорное понятие монад, которое инкапсулирует идеи, считавшиеся ранее недоступными для подобных теорий. 
\end{blockRUS}

\begin{blockENG}
It is difficult to explain the clarity and beauty brought to category theory by people like Daniel Kan\index{Kan, Daniel} and Andr\'{e} Joyal\index{Joyal, Andr\'{e}}. They have each repeatedly extracted the essence of a whole mathematical subject to reveal and formalize a stunningly simple yet extremely powerful pattern of thinking, revolutionizing how mathematics is done.
\end{blockENG}

\begin{blockRUS}
Трудно объяснить ясность и красоту, принесенные в теорию категорий такими людьми, как Дэниэл Кан\index{Кан, Дэниэл} и Андрэ Жуаяль\index{Жуаяль, Андрэ}. Каждый из них последовательно извлекал сущность целых областей математики, чтобы сделать их более явными и формализовать удивительно простые и в то же время чрезвычайно мощные способы мышления, революционизируя само занятие математикой. 
\end{blockRUS}

\begin{blockENG}
All this time, however, category theory was consistently seen by much of the mathematical community as ridiculously abstract. But in the 21st century it has finally come to find healthy respect within the larger community of pure mathematics. It is the language of choice for graduate-level algebra and topology courses, and in my opinion will continue to establish itself as the basic framework in which mathematics is done.
\end{blockENG}

\begin{blockRUS}
Все это время теория категорий последовательно считалась большинством специалистов до нелепости абстрактной. И все же, в 21-м веке она наконец находит здоровое уважение внутри все большего и большего сообщества чистых математиков. Это основной язык курсов по алгебре и топологии магистерского уровня, и, по моему мнению, он будет продолжать становиться тем каркасом, на основе которого делается вся математика. 
\end{blockRUS}

\begin{blockENG}
As mentioned above category theory has branched out into certain areas of science as well. Baez\index{Baez, John} and Dolan\index{Dolan, James} have shown its value in making sense of quantum physics, it is well established in computer science, and it has found proponents in several other fields as well. But to my mind, we are the very beginning of its venture into scientific methodology. Category theory was invented as a bridge and it will continue to serve in that role.
\end{blockENG}

\begin{blockRUS}
Как упоминалось выше, ответвления теории категорий проникли в некоторые другие области науки. Баэз\index{Баэз, Джон} и Долан\index{Долан, Джеймс} показали её важность в придании смысла квантовой физике, она уже основательно закрепилась в теоретической информатике, а также нашла сторонников в нескольких других областях. Тем не менее, по моему мнению, мы находимся только в самом начале ее пути в научную методологию. Теория категорий была создана в качестве связующего звена, и она продолжит исполнять эту роль. 
\end{blockRUS}

\end{document}
