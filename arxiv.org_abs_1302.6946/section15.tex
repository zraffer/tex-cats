\documentclass[CT4S-EN-RU]{subfiles}

\begin{document}

\section{\caseENGRUS{(hidden text)}{ / }{(скрытый текст)}}

\begin{blockENG}
In a front page MIT news article on December 17, 2012 explained that {\em graphs}, systems of vertices and edges are “A simple tool for representing relationships between data, devices or almost anything else has ubiquitous applications in computer science.”\href{http://web.mit.edu/newsoffice/2012/explained-graphs-computer-science-1217.html}. Categories are graphs with one extra piece of structure: the ability to consider two paths, such as two different routes to the grocery store, to be equivalent. Reading it through, the MIT news article would have been more appropriately written about categories, had scientists been adequately aware of them. This is my hope for the course.
\end{blockENG}

\begin{blockRUS}
На первой странице выпуска MIT news за 17 декабря 2012 объясняется, что {\em графы}, системы вершин и ребер, будучи «простым инструментом для представления отношений между данными, устройствами или практически чем-угодно еще, имеют вездесущие приложения в информатике.»\href{http://web.mit.edu/newsoffice/2012/explained-graphs-computer-science-1217.html}. Категории являются графами с еще одним компонентом структуры: возможности считать два различных пути, таких как два различных маршрута к магазину, эквивалентными. Дочитав до конца, я убедился, что, если бы учёные были более адекватно информированы о категориях, данную статью для MIT было бы более уместно написать именно о них. Таковы мои надежды, которые я возлагаю на этот курс. 
\end{blockRUS}

\begin{blockENG}
These communication barriers are seen in businesses when one database “can't talk to” another — the two systems, though ostensibly containing information about the same topic, represent the information differently and therefore cannot be integrated without tremendous cost and overhaul. The barriers can be seen in teaching or training, when an expert has a hard time speaking in a language that the student is ready to understand. Finally, and most relevantly, they can be seen in science. Aside from teaching and training, there is the issue that researchers in even neighboring fields often find the barriers between disciplines to be insurmountable, and much of this has to do with differences in jargon.
\end{blockENG}

\begin{blockRUS}
? 
\end{blockRUS}

\begin{blockENG}
In math this is indeed the case: researchers in even neighboring fields do not understand what each other is doing. But the problem used to be much more pronounced. Category theory, invented in the 1940s has slowly been remedying this problem. Once a problem is formulated categorically it can be analyzed with respect to a growing body of ....
\end{blockENG}

\begin{blockRUS}
? 
\end{blockRUS}

\begin{blockENG}
What mathematics do scientists need to do their work? Calculus and its derivatives, such as differential equations; linear algebra and its span, which includes the likes of tensor analysis — these are invaluable to the process of doing science. In other words, to predicting the outputs of a system based on ones knowledge about it. But the working scientist has more to do than simply predict and understand. He or she needs to organize that the large amounts of information and knowledge, in such a way that it can be apprehended by others. No one is paid to think and understand; people are only paid to think, understand, and {\em communicate their findings}. Category theory is invaluable for the latter.
\end{blockENG}

\begin{blockRUS}
? 
\end{blockRUS}

\begin{blockENG}
But it goes further. If we were perfect thinkers then anything we thought once would be instantly recallable later, but this is not the case. We take notes so we can remember our earlier thoughts at a later time; I often find myself looking through old papers to remember ideas that used to be quite sharp for me, but which now are only vague memories. Thus we need to communicate not only to others but to our later selves. It turns out that if we come up with a good system for doing that, we end up not only using it to communicate to our selves 3 years later, but even 3 seconds later. A good organizational structure fundamentally changes how we do business.
\end{blockENG}

\begin{blockRUS}
? 
\end{blockRUS}

\begin{blockENG}
People often ask “is category theory just a language? Can anything be done with category theory that can't be done without it?” Consider a university like MIT; can the researchers here do anything that they couldn't do by themselves? Isn't it the minds themselves that are so valuable and unique? Perhaps not. Perhaps much of what goes on is enabled and made possible by the organizational structure of the university, which brings people together in the same space with the same rules, providing the  infrastructure necessary to do excellent work. The same is true in category theory: while many of the real advances come from domain-specific creativity, the language and toolset of category theory provides an infrastructure that enriches all involved. Category theory is mankind's most well-oiled conceptual machine to date.
\end{blockENG}

\begin{blockRUS}
Зачастую звучит вопрос: «Является ли теория категорий всего лишь языком? Можно ли сделать что-нибудь при помощи теории категорий, что невозможно сделать без нее?» Рассмотрим университет вроде MIT; может ли исследователь сделать здесь что-либо, что он не смог бы сделать самостоятельно? Не является ли [его] интеллект сам по себе тем, что ценно и уникально? Похоже, что нет. Похоже, большая часть из того, что происходит здесь, обусловлена организационной структурой университета, которая сводит людей в одно время в одном месте с одними и теми же правилами поведения, предлагая инфраструктуру, необходимую для качественной работы. То же верно и для теории категорий: хотя многое из действительно сделанного при помощи нее просходит из творчества в конкретных областях, язык и инструментарий теории категорий предлагает инфраструктуру, которая обогащает все, что ее использует. Теория категорий — это самая хорошо смазанная [т.е. эффективная] понятийная машина на сегодняшний день. 
\end{blockRUS}

\end{document}
